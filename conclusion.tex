\section{Conclusion and Project Experience}
%Possible contributions
The work of this project has the potential to improve program performance by minimizing redundant object reconstruction overhead for data structures. The intra-procedural analysis presented in this project is the core of the analysis, however there is much room for work to analyze the effectiveness of this analysis and transformation technique. The static program analysis concepts relevant  to this project are intra-procedural dataflow analysis, escape analysis, object representatives and the exploring the Soot framework.
%Experience
One of the main challenges of this project was incorporating the inter-procedural analysis needed for dereferencing object pointers to the heap that was essential for this analysis. The usage object representatives were the solution to that problem although it was not an obvious veness of this analysis and transformation technique. The static program analysis concepts relevant solution because purely intra-procedural analysis do not always care about heap references. Secondly the exploration of escape analysis was another major checkpoint in the course of the project because it incorporates some of the key elements of the analysis. However it was fun exploring how escape analysis could be incorporated in the context of this intra-procedural analysis such as this project, because from most paper review about escape analysis it was evident it was most commonly used for inter-procedural analysis and thread safety optimization domains. Lastly and most importantly, the learning curve for Soot was both challenging and rewarding. ``The Soot Survival Guide'' \cite{ref:sootsurvival} by Einarsson and Nielsen was a life saver in the initial stages as well as the blogs from Eric Bodden. It was my first time using a static analysis framework, therefore learning the datflow analysi framework and studying the API library of Soot for hours to find the right commands was great learning experience. It was rewarding because experience in using a robust static analysis tool is great to have and also studying the Soot API helped me understand some of the general features of modern program analysis techniques. Assistance from Derek Rayside was also notable towards progress during the project thanks to his regular constructive feedback and help in forming the key motivational example for this analysis.