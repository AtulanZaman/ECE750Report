\section{Overview of analysis}\label{sec:analysis}
%Mention the two steps. Dataflow analysis step. and then the escape analysis step.
%Insert a pseudocode snippet
For the scope of ECE750, only the intra-procedural analysis of the complete program analysis is completed. This mean that, for this project, the functionality of the program is only limited to doing intra-procedural analysis on a given method to detect whether an immutable local variable or a Field Reference escapes at a certain program point in a method, and checks whether a variable with the same \textit{Object Representative} is used at a later program point. If such a case does happen, then because of the usecase of the immutable object in the respective method, the object cannot be transformed to mutable. First this section explains the intra procedural dataflow analysis that happens in the program. Second it explains the reason for the usage of Object Representatives \cite{ref:or} in the scope of the data flow analysis. Lastly the escape analysis rules used in the project to detect object escaping is explained. The intra-procedural analysis is done using the dataflow analysis API provided by the Soot framework \citep{ref:Soot}.

\subsection{DataFlowAnalysis}\label{sec:dataflow}
The dataflow analysis is a ``may-alias'' forward analysis, where each unit in the analysis graph stores a HashMap of the localvaribles and fieldreferences at the respective program point. Each of the design choices for the data flow analysis are explained below.

\textbf{Direction of Analysis:} The direction of the analysis is chosen to be forward analysis because in each step of the flow through function, the information about the variables in previous program points is of interest to subsequent program points. This is important because in the escape analysis step of the algorithm followed by the dataflow analysis step, the information about the object representatives of previous variables is used to determine whether the it is safe to make the object immutable or not.

%Flowthrough with Object Representative
\textbf{Flowthrough Reaching Condition:} The reaching condition for the flowthrough of information in a unit done according to the following rules. Each unit of the analysis graph stores a HashMap that stores an instance of the Local variable of a field reference along with its Object Representative that is generated along the flowthrough function.
\vspace{5mm}

\begin{algorithm}
\caption{Flowthrough algorithm for dataflow analysis}
\label{fig:flowthrough}
\begin{algorithmic}
  \State Copy InSet to OutSet\;
  \If{Stmt is DefinitionStmt}
	\If{rhs is LocalVariable}
		\State OutSet.put(lhs, rhs.ObjectRep)\;
	\Else
		\State generateObjectRep\;
		\State OutSet.put(lhs, new ObjectRep)\;
	\EndIf	

  \EndIf
\end{algorithmic}
\end{algorithm}
%Mayalias(merge)
\textbf{Merge Condition: } The merge condition for this analysis is done in the form of MayAnalysis where a variable attained different ObjectRepresentatives in different, both the object representative for the object are store with respect to the program point in which the merge is happening. This is necessary because if the object escapes within the branch, and in a subsequent program pointer if either of the ObjectRepresentatives aliases with one of the variables being used in the Stmt, then the object is still considered to be non-transformable to mutable because the used object ``may'' have escaped.

\subsection{Object Representatives}\label{sec:OR}
%Talk about what is object representative (Cite)
In this analysis Object Representatives need to be used because in this analysis objects aliasing with respect to their pointers to the heap are of interest. Generally in order to infer object reference to the heap, an inter-procedural analysis is necessary. However object representatives are a technique of dereferencing heap pointers to local objects in an intra-procedural analysis \cite{ref:or}.

This is evident in our analysis of the client with the escaping object shown previously in figure \ref{fig:client_escape}. In that example, the escaped instance of \texttt{a} would have the same object representative as the field \texttt{escaper} object that is called later in the program although they were assigned to the heap in a seperate method from the subject method. If instead, na integer value assignment for the object instance and the field was used during the dataflow analysis, then this analysis would no be possible because they would have different value assignments. Without dereferencing pointers to the heap there is no way for programs to do this aliasing without using object representatives.
%Talk about why object representative is needed.
%How it is applied in this regard

\subsection{EscapeAnalysis}\label{sec:escape}
%Introduce the concept from the paper (cite)
In this program, escape analysis is used to detect objects that are escaping the stack to heap, which causes objects to be globally accessible and hence prevent tranformation. There has been much work centered around escape analysis in the last decade, most of which has been towards its application in creating thread safe programs and optimizing multithreaded programming. The papers by Choi et al. and Vivien et. al \cite{ref:escapejava}\cite{ref:incrementalescape} were some of the first papers to introduce the escape analysis in the context of inter-procedural analysis. Both the papers were focused on the application of escape analysis to minimize object allocation to the heap with the objective of reducing sychronization operation for threads.

Surprisingly, escape analysis is seldom used in the context of intraprocedural analysis. The only work that relates to this project with respect to escape analysis is its application in dynamic compilation and deoptimization \cite{ref:kotzmann} which uses escape analysis both in the context of intraprocedural analysis interprocedural analysis. Much of the intraprocedural inference techniques used in this work has been inspired by the works presented there.

%Talk about the rules with respect to this analysis (one paragraph each)
There are many programming use cases which causes local objects inside methods to escape the stack and get allocated to the heap, however in the scope of this research only some rules apply to the analysis. The applicable rules of escape analysis that apply in this context are strictly those related to object instance sharing and accessibility. According to \cite{ref:kotzmann} there are two categories of object escaping: \textit{global escape} \& \textit{method escape}. In the context of this research on the global escape of objects is relevant. Objects that suffer from method escape are also called \textit{thread-local} objects which means that the object has escaped the context of the method, but is local with respect to the thread running the method. Since the focus of this analysis is not to optimize synchronization calls with respect to threading, this type of object escape not relavant for this analysis.

The cases of \textit{global escape} relevant to this research is listed below:

\texttt{T.sf = a :} A local variable escapes its local allocation on the stack to the heap when it is assigned to a static field of the class. Being assigned to the static field of a class allow the object instance to be referenced by multiple methods and that is causes the object to escape. In terms of static analysis, this can checked trivially by checking that the left operand is a member of the field deference list and checking that the right operand is a local variable, in which case the assignee is the variable escaping ie.\ \texttt{a}.

%put an algorithm snippet if possible


