\section{Implementation and Evaluation}
%Describe the overall algorithmmen
The implementation of the analysis was completely done with the Soot framework\citep{ref:Soot} for Java bytecode analysis. Below in figure \ref{fig:overall} the wrapper algorithm that uses the escaped analysis and the result of the dataflow analysis is presented.

\begin{algorithm}
\caption{Wrapper algorithm for detect mutable safety}
\label{fig:overall}
\begin{algorithmic}
  \State Do dataflow analysis\;
  \For{Each node \texttt{n} in the dataflow graph}
       \State rhs = Value for Stmt.rhs
       \If{rhs is escaping in n}
           \State flag rhs as ``escaped''
       \EndIf
       \State rhsObjRep = Object Representative for rhs
       \If{rhsObjRep equals Object Representative for existing local in the Map}
		  \If{Any of the aliasing locals are flagged as escaped}
		      \State Add the aliasing tuple to a flagged list
       	  \EndIf
       \EndIf
  \EndFor
  \State Filter the locals from the tuple set
  \State Flag the locals as unsafe to transform
\end{algorithmic}
\end{algorithm}

The analysis below returns a list of tuples from the intra-procedure and escape analysis of a method. The each tuple in the list indicate a set of local variables that have the same object representative pointer to the heap, with atleast one of them escaping and being read by the method at a subsequent program point. The tuples contain not just the locals and fields of the method but also many temporary variables that are introduced by Soot's intermediate form for bytecode analysis. Therefore all the temporary variables are filtered from the tuples sets, and all the Locals from the tuple set are stored in a separate list and this elements of this list indicate which varaibles in this method are unsafe to be transformed to mutable from their immutable declaration.

%Talk about the results of the analysis. %Talk about the limitations of the analysis
The result of the analysis by soot returns a list of variables for each method of the analyzed class that are unsafe to be transformed to mutable. In the bigger scheme of things, what we really interested in is which variables are not in that list because we care more about the variables that can be transformed. For now the analysis has only been tested on toy test cases created during the development of the analysis technique and it has not been run on widely used Java library classes. The limitations of this analysis is that it only addresses the core analysis behind detecting when it is safe to transform immutable object using escape analysis. However for the analysis to be useful, it first needs to infer which objects are immutable to begin with, because we don't care about the variables that are already mutable. Existing research, for example the work of C.Unkel et. al \cite{ref:finalinference} can be used to inference objects with immutable declarations. This would be used to further filter from the list of flagged variables that is returned by this analysis, because currently is returns the list of all variable in a method that are escaping and being read afterwards, whether they are mutable or immutable.

The current analysis returns correct results within the use case domain explored during the development of the analysis. However, the analysis must run on actual useful code in order to validate its utilty.